%documento de clae
\documentclass[a4paper,12pt]{article}

%paquetes
\usepackage[utf8]{inputenc}
\usepackage[spanish]{babel}
\usepackage[total={18cm, 21cm}, top=2cm, left=2cm]{geometry}
\usepackage{amsmath, amssymb, amsfonts, latexsym}
\usepackage{graphicx}
\usepackage{color}
\usepackage{multicol}
\usepackage{nicefrac}

%comandos
\title{Escritura de texto matemático}
\author{Alexis Villavicencio}
\date{\today}

%contenido
\begin{document}
\maketitle
La optimización de funciones no es un tema analizado únicamente con herramientas del cálculo en
una variable y de la programación lineal. Esta se puede generalizar a espacios más generales como
son los espacios de Banach. A continuación se presenta el siguiente problema de optimización:

\begin{align}
   minJ(u,y,a)= \int_{0}^{a} \big(u^{'}(x)\big)^{2} \, dx
\end{align}

\end{document}