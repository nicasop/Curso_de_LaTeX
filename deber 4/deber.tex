%documento de clase
\documentclass[a4paper,12pt]{article}

%paquetes
\usepackage[utf8]{inputenc}
\usepackage[spanish,es-tabla,es-nolayout,es-nodecimaldot]{babel}
\usepackage[total={18cm, 21cm}, top=2cm, left=2cm]{geometry}
\usepackage{amsmath, amssymb, amsfonts, latexsym}
\usepackage{graphicx}
\usepackage[x11names,table]{xcolor}
\usepackage{multicol}
\usepackage{listings}

%comandos
\title{Componer Latex con otros lenguajes de programación}
\author{Alexis Villavicencio}
\date{\today}

%contenido
\begin{document}
\lstset
{
language = C++,
breaklines = true,
basicstyle = \tt\footnotesize ,
keywordstyle = \color{purple},
identifierstyle = \color{yellow},
commentstyle = \color{red},
showstringspaces = false ,
 numbers = left,
 numberstyle = \tiny\color{brown},
 numbersep = 10pt,
 frame = single ,
 rulecolor = \color{teal!20},
 tabsize = 3,
 texcl = true,
 }
 \begin{lstlisting}

#include<stdio.h>
Int main (){
Int n1,n2,resultado=0;
Printf("\t\t\tPROGRAMA PARA SUMAR DOS NUNEROS\n");
Printf("ingrese el primer numero\n");
Scanf ("%d",&n1);
Printf("ingrese el segundo  numero\n");
Scanf ("%d",&n2);
Resultado=n1+n2;
Printf("el resultado de %d + %d = %d",n1,n2,resultado);
Return 0;
}

\end{lstlisting}

\end{document}