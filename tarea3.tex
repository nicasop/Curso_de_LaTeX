%documento de clase
\documentclass[a4paper,12pt]{article}
%paquetes
\usepackage[spanish]{babel}
\usepackage[total={18cm, 21cm}, top=2cm, left=2cm]{geometry}
\usepackage{amsmath, amssymb, amsfonts, latexsym}
\usepackage[utf8]{inputenc}
\usepackage{graphicx}
\usepackage{color}
\usepackage{multicol}
\usepackage{nicefrac}
%comandos
\parindent = 0mm

\author{Alexis Villavicencio}

\title{Escritura de texto matemático}

\date{\today}

%contenido
\begin{document}
\maketitle
La optimización de funciones no es un tema analizado únicamente con herramientas del cálculo en
una variable y de la programación lineal. Esta se puede generalizar a espacios más generales como
son los espacios de Banach. A continuación se presenta el siguiente problema de optimización:
\begin{align}
  &min J(u,y,a)=\int_{0}^{a} \big(u^{'}(x)\big)^{2}\ dx \,+\,\int_{0}  ^{a} \dfrac{a^2}{med(0,a,a^2)} \ dx , \\
\nonumber&sujeta \,a\nonumber 
\end{align}

\begin{align}
 \begin{cases}
-u^{"}(x)+\alpha(x)u(x)=y(x)\quad &en (0,a)\\
u=0 \quad &en \{0,a\}\\
\displaystyle \lim_{x \to a} y(x)=a,\\
a\geq 4
\end{cases}
\end{align}
La idea es optimizar sobre el conjunto de funciones de cada intervalo de la forma [0,a] y determinar
el valor de 
\(
a\geq4 
\)
que indique el mejor intervalo de trabajo.



\end{document}