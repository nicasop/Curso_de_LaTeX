%%%%%%%%%%%%%%%%%%%%%%%%%%%%%%%%%%%%%%%%%

%----------------------------------------------------------------------------------------
%       Clase, paquetes y configuraciones
%----------------------------------------------------------------------------------------

\documentclass[11pt, a4paper]{article} % Font size

\usepackage[utf8]{inputenc}
\usepackage[T1]{fontenc}
\usepackage[spanish,es-nolayout,es-nodecimaldot,es-tabla]{babel}
\usepackage{amsmath}
\usepackage{amsfonts}
\usepackage{amssymb,amsthm}
\usepackage{enumerate}
\usepackage{enumitem}
\usepackage{parskip}
\usepackage{nicefrac}
\usepackage[left=2cm,right=2cm,top=2.5cm,bottom=2cm]{geometry}
\usepackage[colorlinks = true]{hyperref} 


%
\newtheorem{teo}{Teorema}

% Comandos
\newcommand{\R}{\mathbb{R}}
\newcommand{\yds}{\qquad\text{y}\qquad}
\DeclareMathOperator{\proy}{proy}
\DeclareMathOperator{\dd}{d}

\linespread{1.25}

%----------------------------------------------------------------------------------------
%       Datos informativos
%----------------------------------------------------------------------------------------
\title{\begin{large}Álgebra  I\end{large}\\ Formulario de distancias}
\author{Andrés Miniguano T. \\ e-mail: \href{mailto:andres.miniguano@epn.edu.ec}{andres.miniguano@epn.edu.ec} 
   \and Milton Torres E. \\ e-mail: \href{mailto:milton.torres@epn.edu.ec}{milton.torres@epn.edu.ec} }
   
\date{\today}

%contenido
\begin{document}
\maketitle

\abstract{En este documento se presentan las fórmulas de distancia entre el punto, la recta y el plano.}

\section*{Notacion}En lo que sigue usaremos las letras del alfabeto \(a,b,c,\ldots\) para un punto en el espacio con coordenadas dadas por índices; es decir

\[ 
a=  \begin{pmatrix}  a_1 & a_2 & a_3 \end{pmatrix}.
\]

Además, se usarán las letras del alfabeto griego para denotar escalares: \(\alpha, \beta, \gamma, \ldots\). 
\noindent La única excepción a las reglas anteriores se dará con el vector
\[ 
	w=  \begin{pmatrix}  x & y & z\end{pmatrix}
\]
que indican los ejes horizontal, vertical y espacial.

\section*{Definiciones}
\begin{itemize}
\item\bf{Punto:} Es cualquier elemento del espacio, el cual consiste en una tripleta ordenada de números reales; es decir, un elemento de \( \R^3\). Por ejemplo, si \( a\in \R^3\), entonces lo escribiremos como\[a =\left( \begin{matrix}a_1 & a_2 & a_3\end{matrix}\right)\] \\ \item\textbf{Recta:} Dados dos puntos \(a\) y \(b\), una recta con vector director \(a\) y vector constante \(b\) consiste en todos los puntos de la forma\begin{equation*}t \,a + b;\end{equation*}aquí \(t\) es un número real. A esta recta la notamos como\[R: [\langle a\rangle + b].\]
\item\textbf{Plano:} Dados un punto a y un escalar \(\alpha\), un plano de vector normal \(a\) consiste en todos los puntos \(w\) que satisfacen$$a \cdot  w = a_1 x + a_2 y + a_3 z = alpha.$$Aquí \( \cdot\) indica el producto punto entre \(a\) y \(w\); y el plano se nota
\[
H: \, [ \langle a,w \rangle = \alpha ].
\]
\end{itemize}

\end{document}