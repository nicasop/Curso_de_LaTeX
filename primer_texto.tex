%documento de clae
\documentclass[a4paper,12pt]{article}

%paquetes
\usepackage[utf8]{inputenc}
\usepackage[spanish]{babel}
\usepackage[total={18cm, 21cm}, top=2cm, left=2cm]{geometry}
\usepackage{amsmath, amssymb, amsfonts, latexsym}
\usepackage{graphicx}
\usepackage{xcolor}

%comandos
\parindent = 0mm

\author{Alexis Villavicencio}

\title{Hola}

\date{\today}

%contendo
\begin{document}
 \maketitle
  
 {\Large Algebra I} {\large materia maldita}
 
 curso latex
 \section[EPN]{Escuela Politécnica Nacional}
 
 \subsection{La historia de la patata mística}
 \subsubsection{Comienzo}
 \centerline{{\LARGE canci\'{o}n}}
 
 \emph{\textbf{{\Huge \textbackslash Programación}}}
\\[5mm] {\huge\color{gray} probemos un poco del texto en gris}
 \begin{center}
 "No hay mejor lugar como el hogar XD"
 \end{center}
\noindent Los cuatro satélites de Júpiter descubiertos por Galileo son:
\begin{enumerate}
\item Europa
\item Io
\item Ganimedes
 \item Calisto
\end{enumerate}
\end{document}
